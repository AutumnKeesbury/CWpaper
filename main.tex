\documentclass[12pt]{article}
\usepackage[letterpaper, margin=1in, includehead]{geometry}
\usepackage{titling, ragged2e, fancyhdr, titlesec, dirtytalk, setspace, hyperref}
\usepackage{fontspec}
\usepackage{graphicx}
\usepackage[notes, backend=biber]{biblatex-chicago}
\addbibresource{ref.bib}

\newcommand{\thebibsec}

\fancypagestyle{firststyle}
{
   \fancyhf{}
   \renewcommand{\headrulewidth}{0pt} % removes horizontal header line
}

\fancypagestyle{bibstyle}
{
  \fancyhf{}
  \fancyhead[C]{\bfseries \bibname \ \textemdash \ \itshape \thebibsec}
}

\hypersetup{
  colorlinks=true,
  linkcolor=blue,
  filecolor=magenta,
  urlcolor=blue,
  pdftitle={Paper 2},
  breaklinks=true,
  citecolor=black,
}
\urlstyle{same}

\setmainfont{Times New Roman}
\setlength{\headheight}{15pt}
\setlength{\footnotesep}{10pt}

\titleformat{\section}[hang]{\large\bfseries}{\thesection}{10pt}{\RaggedRight}
\titleformat{\subsection}[runin]{\normalsize\bfseries}{(\thesubsection)}{5pt}{}

\pretitle{\RaggedRight\normalsize\bf}
\posttitle{\\ \vspace{12pt}}
\preauthor{\RaggedRight\normalsize}
\postauthor{\\ \vspace{12pt}}
\predate{\RaggedRight\normalsize}
\postdate{\\ \vspace{12pt}}

\title{Generalizing the Bullet: the Transformation of Medicine during the Civil War}
\author{Autumn Keesbury}
\date{2025-26}

\doublespacing

\begin{document}

\newcommand{\thesec}

\maketitle
\thispagestyle{firststyle}
\pagestyle{fancy}
\fancyhf{}
\fancyhead[R]{Keesbury, \thepage}
\fancyhead[L]{\thesec}


\renewcommand{\thesec}{Introduction}
In recent years, a growing body of scholarship has taken for granted the idea that the Civil War marked a great shift in American medicine, and much
of said scholarship has in some way connected this shift to the work of Michel Foucault, with a specific focus on his \citetitle{Foucault1995}.\autocite{Foucault1995}\hspace{1pt}\footnote{See, for example, \citeauthor{Wilson2021} (2021) and \citeauthor{Devine2016} (2016).}
This direction of research is, in fact, quite prominent in areas as distant as Civil War prison archaeology, and has produced several promising results.\footnote{McNutt's recent publications are exemplary here: \citetitle{McNutt2019} (2019), \citetitle{McNutt2019a} (2019), \citetitle{McNutt2021} (2021), \citetitle{McNutt2024} (2024).} However, I feel that these studies have been somewhat incomplete in that, although they
do well to demonstrate the presence of such a shift, they stop just short of explaining the mechanisms by which it was able to occur. This omission is, of
course, understandable, as the effects of the introduction of panopticism into American medical and carceral spaces during the Civil War are so far-reaching
as to merit near infinite exploration. Yet this does not discount the value of investigation into the ways that such an introduction may have taken place,
and indeed to do this would give the existing work on this topic a more solid theoretical underpinning.

But the task of demonstrating a shift in medical perception is not a straightforward one. The fundamental difficulty in this pursuit is thus: no historian
may with complete fidelity see through the eyes of another person, even in a period so recent as the Civil War. The challenge this unfortunate truth poses
is, however, by no means an insurmountable one. For, in the introduction of the 'case' and the move towards a more modern medicine which occurred
in the United States between 1861 and 1865, we are fortunate to find as a primary characteristic of the change an increase in documentation. As shall
be seen later, this is in keeping with the panoptic principle.

The most heavily relied upon source from which this paper draws are the records of the United States Sanitary Commission (USSC), which include letters,
reports, and other \textit{miscellanea bibliorum}. Although my aim here is not to provide a comprehensive history of the Sanitary Commission, although
such a history written in this century is in order, it is necessary to understand the nature of the USSC so as to illustrate properly the reasons
for my reliance on its records.\footnote{For the most complete history of the Sanitary Commission, see Maxwell, \citetitle{maxwell1956lincoln}, also Thompson,
	\citetitle{thompson1956us}; for a history of the Sanitary Commission in the Eastern Theatre, see Mugridge, \citetitle{mugridge1960united}; for a
	history of the Sanitary Commission in the Western Theatre, see Parrish, \citetitle{parrish1990western}; for a more recent, less focused history of the topic of
	death	and dying during the War, see Faust, \citetitle{faust2009republic}.} The Sanitary Commission (active 1861-65) was not, at least initially, a governmental
program. It was through its work during the course of the War that, having started as a civilian organization which operated parallel to official military
and administrative functions in a manner somewhat akin to the modern American Red Cross or the \textit{Médecins Sans Frontières}, the Commission begun to
be integrated into these structures, though never completely losing organizational autonomy.\autocite[cf.][]{novom2020helping}

Although the Sanitary Commission was always distinct from the traditional military-politico structures of organization of command, it did, as shall be
demonstrated later, come to take the shape of these structures, at least in the abstract bureaucratic space with which we shall in part be dealing.
As a result of this, the records of the Commission are invaluable in any such study which concerns itself with the history of institutions. This is, however,
not enough to completely explain my heavy reliance on said records. Indeed, a much less lofty reason takes the place of primacy in this discussion:
in brief, the records of the USSC are potentially the most complete set of primary evidence in this direction of research, and hence they provide an
invaluable look into many seemingly disparate branches of the history of medicine.

With all of this considered, what I wish to accomplish with this paper is to demonstrate that the Civil War marked a great shift in American medicine. I will also
show that this shift was realized in two main ways: (i) structurally, in the ordering of spaces; and (ii) perceptually, in the gaze of the individual doctor. And,
finally, that the result of these changes was the rapid development of a normative medicine which penetrated all the aspects of life which formerly were not subjects
of the medical gaze.

\renewcommand{\thesec}{Early American medical thought}
\section{\thesec} \label{sec:early}
The history of medicine in America, from Independence to the Civil War, is far from a history of institutions. Like many aspects of life in the young republic,
medical care was a largely local, often familial, service. Of course there were cities, and in these cities were hospitals, but these were far from a widespread
phenomenon, and there was slight communication between them. In the late 18\textsuperscript{th} century, the doctor provided treatment at the home of the patient, and
only when they presented an ailment or injury too severe for a family member, often a mother or wife, to tend to. Although formal training was offered, many town
doctors did not receive it, and were at most trained by apprenticing with another doctor, often a relative.\autocite[Although far from offering a complete survey of
	early American medicine,][gives an insightful look into percisely this local nature.]{ulrich1991midwife}

The specific institutions and methods of medicine in the pre-Civil War period are not the focus of this paper, which is instead
concerned with the impact of the Civil War on American medicine. However, a cursory glance at the medical philosophy which then
existed in the United States will be beneficial, if simply to establish a point of reference which elucidates magnitudes of
changes to be examined. The texts selected as guides in this pursuit are George Wallis' 1794 \citetitle{Wallis1794}
and William Yates' 1797 volume \citetitle{yates1797}, which provide valuable insights into the ideas surrounding disease and
medicine with which the authors were contemporary.\autocites{Wallis1794}{yates1797}

These specific works were selected because, when taken together, they contain insights into the two areas of medicine which are
of great interest in this brief look at early American medicine. That their purposes and contents differ somewhat is evident
by simply looking at their titles: for Wallis, the practical and descriptive title \citetitle{Wallis1794} is suited more towards the
country doctor, likely trained through a system of apprenticeship, with Yates' far-reaching and philosophical \citetitle{yates1797}
suggesting a much more rigorous foundation for the students at one of the new nation's few hospitals or medical schools. Furthermore,
Wallis states as the intent of his book \say{[t]o give rational information to those, who, not being properly educated, are obliged to
	practice from necessity}\autocite[\textit{Explanatory Preface}, X]{Wallis1794}. Yates' intended readership, on the other hand, is made clear
from the \say{Advertisement} at the opening of the book: \say{He who abandons Principles in Deference to popular clamour, and he who
	perseveres in Error in spight [sic] of Conviction, may indeed obtain a momentary Celebrity; but they are equally unqualified for the
	Promotion of Science.}\autocite[\textit{Advertisement}]{yates1797}

In comparing these works, it is natural to begin with the definition of the disease, being that upon which all medicine is established. According to Wallis,
\say{[b]y Disease is meant a general or local affection, by which the system is disturbed, or the action of a part impeded, perverted,
	or destroyed}.\footnote{Wallis, 202.} Importantly, Wallis and Yates differ on their interpretation of how fundamental disease is:
Yates asserts that \say{[d]iseases differ form each other, only in the degree of accumulation [withholding of stimuli], or exhaustion of
	the excitability in the whole, or parts of the body}\footnote{Yates, 31.}; whereas Wallis holds that diseases are unique forms
which target individual constitutions or humours.\footnote{Wallis, 58.} The prevalence of taxonomical classification of disease
at the time of the authors suggests that the latter understanding was more widespread, although it should also be noted that these
perspectives are far from mutually exclusive.

Yates' approach to disease yields naturally to a methodology of identification, in that one disease may be distinguished from another
by the \say{the degree of accumulation, or exhaustion of the excitability} presented by the patient.\footnote{Yates, 31.} Wallis similarly
identifies a disease as \say{discovered and distinguished by an enumeration of certain symptoms or appearances with which it is always associated}.\footnote{Wallis, 202.}
Clearly, then, the proper identification of disease was understood to depend upon the accuracy of the doctor's
gaze, with the patient simply being a vessel for the interaction of constitutions and afflictions, with the role of the doctor being then
to identify the truth existing in the workings of the body of the infirm. Indeed, Wallis even goes so far as to say of the idea that
\say{all men are the best judges of their own constitution}: \say{I can by no means allow this to be a truth}.\footnote{\textit{ibid.}, 57.}

The final piece of early American medical philosophy which I wish to identify are the means by which a person was thought to
become afflicted by a disease. Wallis lists three causes of disease: \say{Predisposing -- when the constitution collectively, or in part,
	is in such a situation as is most favorable to produce disease}; \say{Remote, or inducing, which depend on the state of the climate --
	situation mode of life -- indiscretion -- or the elective power of morbid particles, called \textit{miasmata--virus--effluvia}}; and
\say{Proximate or immediate, which are such as from their action constitute the immediate source of the disease}.\footnote{\textit{ibid.}, 202-3.}
Notably, there is no mention of the transmission of disease between two affected bodies, a theme which Yates makes explicit, writing that
\say{[c]ontagion has been enumerated as a cause of pestilential diseases. But as the existence of such a power is by no means provided, it ought
	not to be admitted in philosophical disquisitions}.\footnote{Yates, 33.}

Hence a very basic picture of medical thought in post-Revolutionary America begins to take shape. This medicine was capable of recognizing disease and,
to an extent, classifying them as collections of symptoms which manifested inside of the body of the patient. Yet it was not robust, and did
not have a suitable explanation for the transmission of disease which accounted for the fact that a sick person, when placed into a room which
otherwise would not induce disease in a healthy person, was capable of producing this effect. More fundamentally, however, this philosophy of
medicine did not have the institutional foundation which would allow it to expand and contract with demand: in the late 18th century, there did
not exist a medico-technical apparatus capable of handling, in a swift and effective manner, any event which should require a great volume of
treatment.

\renewcommand{\thesec}{The ordering of spaces}
\section{\thesec} \label{sec:spaces}
A great shift in the way two fundamental medical spaces were ordered occurred during the Civil War. The first was made possible by the
interaction of the medical and the disciplinary, characteristic of the classical epidemic, bringing about the synthesis of a gaze which
penetrated every level of the army. The second was a revolution of the physical space of the hospital, an institution which was not
formerly widespread in the United States, but whose construction was necessitated \textit{en masse} by the War. Taken together, these
two movements, made simultaneously under the unceasing pressure of a truly industrial war, constitute the first part of the transformation
of American medicine which took place between 1861 and 1865.

\subsection{An epidemic of bullets}
The medical history of the Civil War may, at least in some ways, be written as the history of a series of great epidemics which stuck the
United States between 1860 and '65. This portrayal is obviously an oversimplification, but I believe that there is great insight to be found
in studying at least the evolution of medical institutions during the War through this lens. Indeed, when understood as responding to epidemics,
many of the great re-orderings which the medical field underwent during the War years become much easier to see. It is this simple shift in
viewpoint which, by virtue of its ability to provide far-reaching insights, served as the initial impetus for this paper.

Before arriving at these insights, however, it must be understood what exactly is meant by the term 'epidemic'. In \citetitle{Foucault1994},
Michel Foucault describes an epidemic as being \say{more than a particular form of a disease ... it was an autonomous, coherent, and adequate
	evaluation of disease}.\autocite[23]{Foucault1994} Thus an epidemic is understood not simply by the symptoms through which the disease
manifested itself, but also through its place in a social body. This separation of a disease from the direct physical form of its subject
is a weaker form of a process which shall be discussed in some detail at a later point: the identification of patient and disease
as fundamentally distinct entities. For the moment, however, it is important only to understand that the epidemic was a disease which was
tied to various conditions; as Foucault puts it, the \say{essential basis is determined by the time, the place, the 'fresh, sharp, subtle,
	penetrating' air of Nîmes in winter or the sticky, thick, putrid air of Paris during a long, heavy summer}.\footnote{\textit{ibid.}}

An organizational structure which facilitated the synthesis of a homogeneous picture of an epidemic through superimposing and cross-checking
medical gazes was present in the Armies of the United States during the American Civil War, at all levels of administrative functioning.
Regimental surgeons and hospitals were tasked with documenting individual cases, meteorological data, and various other information situated at
a similar level. Then there were the surgeons of the general hospitals, tasked with investigating and recording extraordinary cases which may be
exemplary, but also with conducting research into the nature of various conditions which were difficult or time-consuming to treat.\footnote{Such research is famously
	the subject of S. Weir Mitchell's postwar literature (cf. Lisa Herschbach's excellent article \citetitle{Herschbach1995}).} Lastly,
situated near the highest administrative level, were the doctors who worked directly for the United States Sanitary Commission (USSC),
and whose duty it was to  visit and write high-level
reports on the conditions of hospitals under the Sanitary Commission’s control. Taken together, the USSC encompassed four parallel, unlimited series
which extend the domain of medical perception infinitely: the study of topographies (conducted at the top level by doctors like S.B. Hunt)
\autocite{ussc:6:882}, meteorological observations (like those collected by Lyman)\autocite{ussc:6:775}, monitoring epidemics (see the reports on
outbreaks in hospitals)\autocite{ussc:10:1523}, and the description of extraordinary cases (e.g. Howard’s report on a case of death during the
administration of chloroform).\autocite{ussc:9:1340}

It is these parallel gazes which, through their integration into the military-political structures of the US Army, were able to cover a
domain which was in many respects broader than that of the institutions to which they were formally subordinate. Indeed, everywhere the Army
went, it was followed by, or moved in lockstep with, the ever-vigilant gaze of a medicine of epidemics. Furthermore, there is often not just an isomorphism
between the realized structures of the military and medical, but a whole series of such correspondences between their possible configurations: that is to say, a
reorganization of one institution is accompanied by a similar reorganization of the other. In this sense, the medical gaze which knows an epidemic forms a negative
copy of the individualizing disciplinary gaze; what the positively-determined structures of the military do not cover, the medical gaze steps in to observe.
Through the union of these two institutions, a complete and totally penetrating gaze is inscribed upon a space of disease and disorder which is opposed, at
each point, to their very workings.

There is no better example of this gaze than the camp inspection. Conducted by the Sanitary Commission, these inspections were far from
strictly 'medical', and in this aspect they demonstrate the great expansion of the domain of medicine. The choice
of camp inspection form to examine is somewhat arbitrary, as they all list the same questions for the inspector to answer, so I will be looking
at one performed on the camp of the 21st Ohio Volunteer Infantry Regiment at Camp Jefferson, KY, on 9 January 1862.\autocite{ussc:20:699}
\footnote{The reason for my choosing this specific camp is quite personal: James Keesbury, a distant relative of mine, served as a
	private in company K of the 21st OH Volunteers.} Alongside what one might expect to be in such an inspection (cleanliness of the men, number of
infirm, assessment of the medical officers, ventilation of the tents, conformity with regulations, etc.), one finds detailed assessments of
the \say{character of the camp site}: the situation of the camp (the 21st OH was camped on a plain), the amount of shade (their camp was unshaded),
the direction of the prevailing wind (S.W.), the character of the soil (loose loam), and of the subsoil (firm loam). There is also discussion of the
condition of the troops which is not strictly sanitary or medical; one question even concerns whether or not \say{they take pride} in their regulation
uniforms.\footnote{I wish here to make a brief note of something about these inspection forms which occurred to me during the revision process: many questions are not answered properly,
	if at all. A good example of this is that of whether or not the men take pride in their uniforms. Upon first writing this paper, I did not think this
	detail important enough to merit its inclusion, however I now am of a different mind. Indeed, the question of why, exactly, such questions so often remained
	blank is one which is potentially significant in evaluating the validity of the argument I present in this section.}

All of this is to say that, even as early as by the winter of 1861-62, a systematic and meticulous medico-military gaze was present at all levels.
This gaze, by being both disciplinary and medical, was able to do something much more profound than maintaining order and health in the army: it was
able to split the military unity into two distinct spaces --- one strictly disciplinary, the other, its negative, strictly medical --- in which nothing
went unobserved. Moreover, this split was duplicated at each administrative level, from the military district down to the individual regiment.

\subsection{The panoptic hospital}
At the same time as the restructuring of the hygienic gaze just explored, a much more physical space was undergoing a transformation of a no less
fundamental nature. In the General Hospital, the class of medical institution at the highest administrative level during the Civil War, are to be found,
alongside explicitly medical aspects, a number of features which bear striking resemblance to those found in the prisons of the era. I argue that these
similarities are not mere happenstance, but rather are a product of systematic coincidences in the goals of disciplinary institutions and the medical
institutions of the Civil War.

These coincidences arise from the very nature of the hospital as a space of observation. Figure \ref{fig:SedgwickHosp} illustrates exactly
this: the arrangement of wards perpendicular to the tangent of the circle whose circumference is a covered walkway is exactly that which optimizes the ability
for a minimal number of staff to observe a maximal number of patients. Along with the overall geometry of the hospital, the individual ward is configured
in a way which is \say{convenient for assigning the different assistants and attendants to their duties}.\autocite[919]{MedHist_1-3}
The hospital, then, was an intensely spatial endeavor: an institution built on the principle of an \say{[e]conomy of space}, whose most basic purpose
was not to treat, but to observe.\autocite[935]{MedHist_1-3}

It was also the goal of the hospital to fashion from the plurality of diseases and bodies an ordered space in which a doctor may identify an individual
by not just his name, regiment, company, and rank, but also by a list of symptoms, injuries, and prior treatment; put briefly, to create from each
man a history of which he alone was the subject. The ward, then, becomes not simply a space, but a process by which to each point of the
\(187 \times 24\)-foot rectangle was assigned a name and its accompanying history. From the great mass of those wounded and sick, the ward
forms a collection of completely knowable sets, the union of which is not simply the original unstructured mass, but rather a table,
as suggested by Figure \ref{fig:ward}.

Much as the goal of the prison is not eliminating crime, but rather \say{producing delinquency, a specific type, a politically or economically
	less dangerous [...] form of illegality}, the goal of the hospital is not to eliminate disease, but to structure it -- to create \say{complex spaces that
	are at once architectural, functional, and hierarchical}.\autocite[277; 148]{Foucault1995} It is also similar to the prison, the school, or the
factory in that, just as these institutions create knowledge of their inhabitants by exercising a disciplinary power, the hospital is an institution
which makes use of the doctor-patient power relation to make the patient at once the subject of knowledge and its object.\footnote{Foucault, \textit{Discipline and punish}, 204.}
The general hospital, then, was a panoptic construction, built with the express intention of surveillance both constant and minimal by the
partitioning and classifying gaze of the doctor.


\begin{figure}[htp]
	\centering
	\includegraphics[width=0.70\textwidth]{graphics/Medical_and_surgical_history_pg946.png}
	\caption{
		\textit{Sedgwick Hospital, Greenville, LA. -- scale 120 feet to the inch: 1, wards; 2, Administration building; 3, Guard-house, knapsack-room,
			and store-house; 4, Dining-rooms; 5, Kitchen; 6, Cistern; 7, Covered ways through which a hallway runs with hand-cars for carrying food to the wards.}
		(Image and caption reproduced from \citetitle[946]{MedHist_1-3})
	}\label{fig:SedgwickHosp}
\end{figure}

\begin{figure}[htp]
	\centering
	\includegraphics[width=0.70\textwidth]{graphics/Medical_and_surgical_history_pg944.png}
	\caption{
		\textit{Plan for a ward at a general hospital, issued by Secretary of War Stanton on 20 July, 1864.}
		(Image reproduced from \citetitle[944]{MedHist_1-3})
	}\label{fig:ward}
\end{figure}


\newpage
\renewcommand{\thesec}{The creation of the case}
\section{\thesec} \label{sec:patient}
The first of the great shifts in American medicine which the Civil War induced was the rearrangement of spaces, namely the disciplinary space of the army
and the physical space of the hospital. There was, however, at the same time a much subtler shift, born not of grand spaces or militant
discipline, but rather formed at the very point of genesis of all medicine: the interaction of the doctor and patient. It was not a revolution in
the sense that it moved great bodies of troops, but in that it fundamentally changed the nature of medical perception. This is a history of how,
during the Civil War, the once strictly remedial gaze of the surgeon, the doctor, and the pharmacist took on a normative character.

\subsection{A generalized injury}
A bullet wound is the most easily knowable disease.\footnote{In the sense of Yates and, to some extent, of Wallis.}
It has not to do with the mysterious interactions of the body's infinitely complex systems
at a level so minute as to be nearly invisible, but with the action of a foreign object with which every person in the army was intimately familiar:
the bullet. Furthermore, the list of ways which a soldier is most commonly struck by a bullet is very easily enumerated, and with less discretion required
of the surgeon than, say, a list of symptoms for a disease of the lungs. Such injuries do not present themselves so differently as to be nearly unrecognizable
from case to case, and it is no great difficulty to classify them according to their nature.

The work of the doctor, then, becomes not to treat according to the multitude of symptoms and the temperament of the patient, but to remove, as far as possible,
the patient from consideration. This is a form of the disciplinary method described by Foucault as such: \say{The examination, surrounded by all its documentary
	techniques, makes each individual a 'case': a case which at one and the same time constitutes an object for a branch of knowledge and a hold for a branch of power}
\autocite[191]{Foucault1995}. In removing as much specificity as possible, the true essence of the affliction is discovered a sort of average of cases. In order
to treat one case, then, the doctor must hold in his mind the plurality of cases. The magnitude of the shift from Wallis' unique forms of disease to this generalized
medical perception is remarkable.\autocite[58]{Wallis1794}

Nowhere in the records of the USSC is a better example of this to be found than in a report by Dr. Ira Russell on various sorts of common injuries experienced by soldiers. Specifically in
reference to injury of the knee joint, \say{the history of one of these cases}, he remarks, \say{is the history of all}\autocite[34]{ussc:2:194}. Dr. Russell
describes the initial state of the patient as being \say{very comfortable with slight local pain and swelling}. The patient would experience no
\say{constitutional disturbances} for as many as two days, however they would appear by the third or fourth day, immediately following an increase in
\say{pain and swelling of the knee}. Russell describes how, at the battle of Prairie Grove, amputations were not performed on those whose knee joints were
wounded, resulting in \say{severe} local symptoms and a fever that \say{greatly reduced the patient}. This is used to stress the importance of timely
amputation in such cases, as Russell notes that, when the surgeons did finally amputate the patient's leg, they did so \say{without in any way improving the
	chance of recovery}.

Only after removing the patient, then, may the symptoms be recognized, interpreted, and synthesized into a diagnosis: it is this that is the power of the hospital.
By interacting so often with the bodily injury, the doctor is conditioned to think of his work in terms of this generalizing perspective. The disease, once
treated as something which depended greatly upon the temperament of the patient, now becomes little more than a generalized form of injury.

There is also something to be said of the documentary process which allowed for this transformation of disease into injury. By the collection of great quantities
of data, a whole series of patterns which were formerly obscured by small sample-sizes now came into a sharper focus than was ever possible. These data, however,
could not be interpreted unless one concession was made by the doctor: that a disease is fully expressed in its symptoms. Foucault describes this process as follows:
\say{To the exhaustive presence of the disease in its symptoms corresponds the unobstructed transparency of the pathological being with the syntax of a descriptive
	language: a fundamental isomorphism of the structure of the disease and of the verbal form that circumscribes it}\autocite[95]{Foucault1994}.

\section{Conclusion}
The American Civil War was a transformative time for medical perception within the United States. A large part of this is due to the sheer volume of injuries
and illnesses, and their concentration in space and time, which result from an industrial war. I have attempted to outline the process by which this medical
war gave rise to a normative and modern medicine, and how this change was manifested in the very space of medicine. This I feel I have done adequately.

I do, however, feel that there is significant room for further work on this topic, and indeed I am not nearly done with this area of research. There are
a couple of places I feel work was limited, the foremost of which is my over-reliance on the records of the USSC. Although they are valuable, they do not
provide a complete picture of the medicine of the time, and this is something which I realized far too late into my work. Moreover, I think that the two main
sections of this paper, that which covers spaces and that which covers the medical gaze, could have been their own papers, and I do plan to write more extensively
about each individually in the future. This work was further limited in that I had little room to mention the roles of race and gender, both of which are important in
any discussion of the Sanitary Commission and Civil War medicine.\autocite{schwalm2018body} This is another major consideration in any further work on this topic.

I recognize that the scope of this work was somewhat overloaded, and that the course of a semester was not nearly long enough to write a thorough history of the
transformation of Civil War medicine. What I have done is, I hope, provided a point of entry for future research into this approach to the subject which I believe
is capable of yielding important results in future.

\newpage
\pagestyle{bibstyle}
\nocite{*}
\renewcommand{\thebibsec}{Primary Documents}
\printbibliography[keyword={primary}, title=\thebibsec]
\renewcommand{\thebibsec}{References}
\printbibliography[keyword={secondary}, title=\thebibsec]

\end{document}
