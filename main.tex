\documentclass[12pt]{article}
\usepackage[letterpaper, margin=1in, includehead]{geometry}
\usepackage{titling, ragged2e, fancyhdr, titlesec, dirtytalk, setspace, hyperref}
\usepackage{fontspec}
\usepackage[notes, backend=biber]{biblatex-chicago}
\addbibresource{ref.bib}

\newcommand{\thebibsec}

\fancypagestyle{firststyle}
{
   \fancyhf{}
   \renewcommand{\headrulewidth}{0pt} % removes horizontal header line
}

\fancypagestyle{bibstyle}
{
  \fancyhf{}
  \fancyhead[C]{\bfseries \bibname \ \textemdash \ \itshape \thebibsec}
}

\hypersetup{
  colorlinks=true,
  linkcolor=blue,
  filecolor=magenta,
  urlcolor=blue,
  pdftitle={Paper 2},
  breaklinks=true,
  citecolor=black,
}
\urlstyle{same}

\setmainfont{Times New Roman}
\setlength{\headheight}{15pt}
\setlength{\footnotesep}{10pt}

\titleformat{\section}[hang]{\large\bfseries}{\thesection}{10pt}{\RaggedRight}
\titleformat{\subsection}[runin]{\normalsize\bfseries}{(\thesubsection)}{5pt}{}

\pretitle{\RaggedRight\normalsize\itshape}
\posttitle{\\ \vspace{14pt}}
\preauthor{\RaggedRight\normalsize}
\postauthor{\\ \vspace{14pt}}
\predate{\RaggedRight\normalsize}
\postdate{\\ \vspace{14pt}}

\title{Dark Woke Essay for the William M. Locke Prize}
\author{Professor Stephen Andrews 
\\\vspace{14pt} Autumn Keesbury}
\date{HIST-A 348}

\doublespacing

\begin{document}

\newcommand{\thesec}

\maketitle
\thispagestyle{firststyle}
\pagestyle{fancy}
\fancyhf{}
\fancyhead[R]{Keesbury, \thepage}
\fancyhead[L]{\thesec}

Thesis: \textit{The Civil war marked a great shift in American medicine. This shift was realized in two main ways: (i) structurally, in
	the ordering of spaces; and (ii) perceptually, in the gaze of the individual doctor. The result of these changes was a rapid development
	of a normative medicine which penetrated all the aspects of life which formerly were not the subject of medicine.}

\renewcommand{\thesec}{Early American medical thought}
\section{\thesec} \label{sec:early}
The history of medicine in America, from Independence to the Civil War, is far from a history of institutions. Like many
aspects of life in the young republic, medical care was a largely local, often familial, service. Of course there were
cities, and these cities had hospitals, but these were far from a widespread phenomenon, and there was slight communication
between them. The doctor provided treatment at the home of the patient, and only when they presented an ailment or injury too
severe for a family member, often a mother or wife, to tend to. Although formal training was offered, many town doctors did not
receive it, and were at most trained by apprenticing with another doctor, often a relative.

The specific institutions and methods of medicine in the pre-Civil War period are not the focus of this paper, which is instead
concerned with the impact of the Civil War on American medicine. However, a cursory glance at the medical philosophy which then
existed in the United States will be beneficial, if simply to establish a point of reference which elucidates magnitudes of
changes to be examined. The specific texts selected as references for this section are George Wallis' 1794 \citetitle{Wallis1794}
and William Yates' 1797 volume \citetitle{yates1797}, which provide valuable insights into the ideas surrounding disease and
medicine with which the authors were contemporary.

It is natural to begin with the definition of the disease, being that upon which all medicine is established. According to Wallis,
\say{[b]y Disease is meant a general or local affection, by which the system is disturbed, or the action of a part impeded, perverted,
	or destroyed}\autocite[202]{Wallis1794}. Importantly, Wallis and Yates differ on their interpretation of how fundamental disease is:
Yates asserts that \say{[d]iseases differ form each other, only in the degree of accumulation [withholding of stimuli], or exhaustion of
	the excitability in the whole, or parts of the body}\autocite[31]{yates1797}; whereas Wallis holds that diseases are unique forms
which target individual constitutions or humours\autocite[58]{Wallis1794}. The prevalence of taxonomical classification of disease
at the time of the authors suggests that the latter understanding was more widespread, although it should also be noted that the
distance which separates these two views is far from irreconcilable.

Yates' approach to disease yields naturally to a methodology of identification, in that one disease may be distinguished from another
by the \say{the degree of accumulation, or exhaustion of the excitability} presented by the patient \autocite[31]{yates1797}. Wallis similarly
identifies a disease as \say{discovered and distinguished by an enumeration of certain symptoms or appearances with which it is always associated}
\autocite[202]{Wallis1794}. Clearly, then, the proper identification of disease was understood to depend upon the accuracy of the doctor's
gaze, with the patient being simply a vessel for the interaction of constitutions and afflictions, with the role of the doctor being then
to identify the truth existing in the workings of the body of the infirm. Indeed, Wallis even goes so far as to say of the idea that
\say{all men are the best judges of their own constitution}: \say{I can by no means allow this to be a truth}\autocite[57]{Wallis1794}.

The final piece of early American medical philosophy which I wish to identify are the means by which a person was thought to
become afflicted by a disease. Wallis lists three causes of disease: \say{Predisposing -- when the constitution collectively, or in part,
	is in such a situation as is most favorable to produce disease}; \say{Remote, or inducing, which depend on the state of the climate --
	situation mode of life -- indiscretion -- or the elective power of morbid particles, called \textit{miasmata--virus--effluvia}}; and
\say{Proximate or immediate, which are such as from their action constitute the immediate source of the disease}\autocite[202-3]{Wallis1794}.
Notably, there is no mention of the transmission of disease between two infected bodies, a theme which Yates makes explicit, writing that
\say{[c]ontagion has been enumerated as a cause of pestilential diseases. But as the existence of such a power is by no means provided, it ought
	not to be admitted in philosophical disquisitions}\autocite[33]{yates1797}.

Hence a very basic picture of medical thought in post-Revolutionary America is formed. This medicine was capable of recognizing disease and,
to an extent, classifying them as collections of symptoms which manifested inside of the body of the patient. Yet it was not robust, and did
not have a suitable explanation for the transmission of disease which accounted for the fact that a sick person, when placed into a room which
otherwise would not induce disease in a healthy person, was capable of producing this effect. More fundamentally, however, this philosophy of
medicine did not have the institutional foundation which would allow it to expand and contract with demand: in the late 18th century, there did
not exist a medico-technical apparatus capable of handling, in a swift and effective manner, any event which should require a great volume of
treatment.

\renewcommand{\thesec}{The ordering of spaces}
\section{\thesec} \label{sec:spaces}
\subsection{An epidemic of bullets}
The medical history of the Civil War may, at least in some ways, be written as the history of a series of great epidemics which stuck the
United States between 1860 and '65. This portrayal is obviously an oversimplification, but I believe that there is great insight to be found
in studying at least the evolution of medical institutions during the War through this lens. Indeed, when understood as responding to epidemics,
many of the great re-orderings which the medical field underwent during the War years become much easier to see. It is this simple shift in
viewpoint which, by virtue of its ability to provide far-reaching insights, served as the initial impetus for this paper.

Before arriving at these insights, however, it must be understood what exactly is meant by the term 'epidemic'. In \citetitle{Foucault1994},
Michel Foucault describes an epidemic as being \say{more than a particular form of a disease ... it was an autonomous, coherent, and adequate
	evaluation of disease}\autocite[23]{Foucault1994}. Thus an epidemic is understood not simply by the symptoms through which the disease
manifested itself, but also through its place in a social body. This separation of a disease from the direct physical form of its subject
is a weaker form of a process which shall be discussed in some detail in Section \ref{sec:patient}: the identification of patient and disease
as fundamentally distinct entities. For this section, however, it is only important to understand that the epidemic was a disease which was
tied to various conditions; as Foucault puts it, the \say{essential basis is determined by the time, the place, the 'fresh, sharp, subtle,
	penetrating' air of Nîmes in winter or the sticky, thick, putrid air of Paris during a long, heavy summer}\footnote{\textit{ibid.}, 23}.

An organizational structure which facilitated the synthesis of a homogeneous picture of an epidemic through superimposing and cross-checking
medical gazes was present in the Armies of the United States during the American Civil War, at all levels of administrative functioning.
Regimental surgeons and hospitals were tasked with documenting individual cases, meteorological data, and various other information situated at
a similar level. Then there were the surgeons of the general hospitals, tasked with investigating and recording extraordinary cases which may be
exemplary, but also with conducting research into the nature of various conditions which were difficult or time-consuming to treat. Lastly,
situated near the highest administrative level, were the doctors who worked directly for the USSC, and whose duty it was to visit and write high-level
reports on the conditions of hospitals under the Sanitary Commission’s control. Taken together, the USSC encompasses four parallel, unlimited series
which extend the space of medical knowledge infinitely: the study of topographies (conducted at the top level by doctors like S.B. Hunt)
\autocite{ussc:6:882}, meteorological observations (like those collected by Lyman)\autocite{ussc:6:775}, monitoring epidemics (see the reports on
outbreaks in hospitals)\autocite{ussc:10:1523}, and the description of extraordinary cases (e.g. Howard’s report on a case of death during the
administration of chloroform)\autocite{ussc:9:1340}.

It is these parallel gazes which, through their integration into the military-political structures of the Armies of the United States, were able to cover a
domain which was in many respects broader than that of the institutions to which they were formally subordinate. Indeed, everywhere the Armies of the Union
went, they were followed by, or moved in lockstep with, the ever-vigilant gaze of a medicine of epidemics. Furthermore, there is often not just an isomorphism
between the realized structures of the military and medical, but a whole series of such correspondences between their possible configurations: that is to say, a
reorganization of one institution is accompanied by a similar reorganization of the other. In this sense, the medical gaze which knows an epidemic forms a negative
copy of the individualizing disciplinary gaze; what the positively-determined structures of the military do not cover, the medical gaze steps in to observe.
Through the union of these two institutions, a complete and totally penetrating gaze is inscribed upon a space which is necessarily opposed, at each point, to
their very workings.


\subsection{The panoptic hospital}

\renewcommand{\thesec}{The creation of the case}
\section{\thesec} \label{sec:patient}

\newpage
\pagestyle{bibstyle}
%\nocite{*}
\renewcommand{\thebibsec}{Primary Documents}
\printbibliography[keyword={primary}, title=\thebibsec]
\renewcommand{\thebibsec}{References}
\printbibliography[keyword={secondary}, title=\thebibsec]

\end{document}
