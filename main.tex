\documentclass[12pt]{article}
\usepackage[letterpaper, margin=1in, includehead]{geometry}
\usepackage{titling, ragged2e, fancyhdr, titlesec, dirtytalk, setspace}
\usepackage{fontspec}
\usepackage[notes, backend=biber]{biblatex-chicago}
\addbibresource{ref.bib}

\newcommand{\thebibsec}

\fancypagestyle{firststyle}
{
   \fancyhf{}
   \renewcommand{\headrulewidth}{0pt} % removes horizontal header line
}

\fancypagestyle{bibstyle}
{
  \fancyhf{}
  \fancyhead[C]{\bfseries \bibname \ \textemdash \ \itshape \thebibsec}
}

\setmainfont{Times New Roman}
\setlength{\headheight}{15pt}
\setlength{\footnotesep}{10pt}

\titleformat{\section}[hang]{\normalsize\bfseries}{\thesection}{10pt}{\RaggedRight}

\pretitle{\RaggedRight\normalsize\itshape}
\posttitle{\\ \vspace{14pt}}
\preauthor{\RaggedRight\normalsize}
\postauthor{\\ \vspace{14pt}}
\predate{\RaggedRight\normalsize}
\postdate{\\ \vspace{14pt}}

\title{Dark Woke Essay for the William M. Locke Prize}
\author{Professor Stephen Andrews 
\\\vspace{14pt} Autumn Keesbury}
\date{HIST-A 348}

%% \doublespacing

\begin{document}
\maketitle
\thispagestyle{firststyle}
\pagestyle{fancy}
\fancyhf{}
\fancyhead[R]{Keesbury, \thepage}
\fancyhead[L]{\thetitle}

\say{I go a great deal into the Hospitals. Washington is full of them \textemdash both in town and out around the outskirts}, wrote Walt Whitman in January of
1863\autocite{Whitman1863}. In these hospitals, Whitman found \say{the best expression of American character I have ever seen or conceived} exhibited by
\say{these ranks of sick and dying young men}. Whitman was not alone in perceiving a truth of the American Civil War, and indeed of America in general,
within the hospitals of his day; in fact, many of his contemporaries, perhaps most famously S. Weir Mitchell, found themselves returning to the scene of the
surgical ward even decades after Appomattox, in search of the War and its meaning\autocite{Herschbach1995}. Although these men concerned themselves mostly
with the poetry of the medical, it remains true that the Civil War was a heavily medical and medicalized event.

The Civil War of course saw a great increase in the technical ability of medicine -- by what means and how effectively a doctor was able to diagnose and treat a patient -- 
but it also saw a far more fundamental restructuring of American medicine. For, in the direct prelude to the War, medicine was greatly localized and treatment was 
generally delivered by a town doctor, typically within one's home; whereas in the days immediately following the War, one sees a much more centralized and 
bureaucratic medical system, spread across the country, which would generally treat patients exclusively within the walls of the hospital.

It is this transformation, its mechanisms, its causes, and its greater bearing on the history of American medical science, which I will investigate in this paper.
I will draw upon the extensive records of the United States Sanitary Commission, an extra-governmental organization which operated similarly to an Executive Department
concerned with hygiene and medical practices, as well as a vast body of research on the theory and practice of medical history, often with specific reference to
the American Civil War. I shall argue that understanding the War as a medical event, and specifically analyzing it as an epidemic\footnotemark, yields many insights
into how and why this change took place. All of this shall be to argue the conclusion that, during the Civil War, the bureaucratic and technical aspects of medicine
were separated, and thus were allowed to change in their natures: the former turned into a great disciplinary institution, the latter into a more normative, less
individual, procedure.

\newpage
\pagestyle{bibstyle}
\nocite{*}
\renewcommand{\thebibsec}{Primary Documents}
\printbibliography[keyword={primary}, title=\thebibsec]
\renewcommand{\thebibsec}{References}
\printbibliography[keyword={secondary}, title=\thebibsec]

\end{document}
